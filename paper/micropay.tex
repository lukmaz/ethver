% !TEX root = ethver.tex

\section{Case study: Micropay protocol}

\subsection{Original contract code}

\lstinputlisting[caption=Contract code for the original Micropay protocol, label=lst:micro_v1]{../examples/micro_v1.etv}

For this protocol we want to verify the following property: \emph{whenever a user U issues a payment, 
the merchant M gets paid with probability $1/N$}.
This translates to the PRISM properties\footnote{Recall that for non-deterministic models in PRISM we must use 
\lstinline{Pmin}/\lstinline{Pmax} operator which computes minimum/maximum probability over all the non-deterministic
choices in the model.}:
\[
\begin{tabular}{c}
\lstinline{Pmin=? [F endM & balance1 >= 1]} \\
\lstinline{Pmax=? [F endM & balance1 >= 1]}
\end{tabular}
\]

These two properties correspond to the \emph{honest mode} --- if both players follow the protocol, then this probability
should be equal to $1/N$.
However, besides the \emph{honest mode} verification, we would like to verify if
\begin{itemize}
\item if merchant M is malicious, then he cannot increase his chances of getting money.
\item if user U is malicious, then he cannot decrease the probability of getting money by merchant M,
\end{itemize}
This leads to the total of 4 PRISM properties listed in the table below
(to reduce computational complexity, we perform all the computations for $N = 2$):

\[
\begin{tabular}{|c|c|c|c|}
\hline
property & adversary & target & \shortstack{result} \\
\hline
\lstinline{Pmin=? [F endM & balance1 >= 1]} & None & 1/2 & 1/2 \\
\hline
\lstinline{Pmax=? [F endM & balance1 >= 1]} & None & 1/2 & 1/2 \\
\hline
\lstinline{Pmax=? [F balance1 >= 1]} & M & 1/2 & 1/2 \\
\hline
\lstinline{Pmax=? [F endM & has_revealedM & balance1 < 1]} & U & 1/2 & \textbf{1} \\
\hline
\end{tabular}
\]
Note that in the third property, we have negated the condition: instead of computing \emph{the minimal probability
that M reveals his winning ticket and gets paid}, we are computing \emph{the maximal probability, that M reveals
his winning ticket and does not get paid}.
This is only because the second property is easier expressible in PRISM properties language.

After verifying these properties in PRISM it turns out that the first three properties 
evaluate to the correct value of $1/2$, however the last one evaluates to 1.
In order to investigate the problem, we will use the method known from the previous section for generating
a witness for unwanted path in PRISM.
We will use the following reachability property:
\lstinputlisting{../examples/micro_witness.props}
This property can be read as \emph{merchant M finishes with \lstinline{balance1 = 0} despite ha has revealed
his winning ticket}.

After verifying this property in PRISM, it turns out, the property is true and moreover PRISM provides us the 
following interleaving for the witness:
\begin{itemize}
\item merchant M executes \lstinline{payment_request}
\item user U executes \lstinline{payment_issuance}
\item merchant M executes \lstinline{open}
\item user U calls \lstinline{release} contract function
\item merchant M calls \lstinline{release} contract function
\end{itemize}
This illustrates a real flaw in the protocol: a front-running attack.
In this attack a malicious user U can produce a winning ticket by simulating
the merchant protocol and broadcast his version of \lstinline{release} transaction which can be processed 
before the \lstinline{release} transaction created by merchant M.
Moreover, user U can monitor the network and wait until M's transaction appears in the network and only then
react with broadcasting his own transaction.
In such case there are two concurrent transactions in the network and among them only one can be confirmed.
Which transaction will be chosen depends on several random factors.
In fact, if U has good enough network setup, he can force his transaction to be accepted before the M's one
with high probability.

\subsection{Mitigating the front-running attack}

The front-running attack can be mitigated using a penalty deposit.
At the beginning of the protocol the user U deposits some amount of money (larger than the amount of payment).
This deposit can be released in one of the two following ways:
\begin{itemize}
\item after some time $\Delta t$ the deposit goes back to U,
\item the deposit is burned if one provides two competitive winning tickets.
\end{itemize}
In case user U tries to perform a front-running attack, the merchant M will have the knowledge of two 
competitive winning tickets and he could use the \emph{burn} function to penalize U.
If U honestly follows the protocol, he can get the deposit back after the appropriate time period has passed.
The complete code of the Micropay contract updated with this deposit penalty mechanism is depicted in listing
\ref{lst:micro_v2}.

\lstinputlisting[caption=Micropay v2, label=lst:micro_v2]{../examples/micro_v2.etv}

In order to check if this contract is resilient to the front-running attack, we must slightly modify the properties:
now in case of honest execution of the protocol the user U finishes with \lstinline{balance0 = 2} (we set the 
deposit to 2).
In case of a successful attack, the merchant M does not receive his payment, however then U will end up 
with \lstinline{balance0 = 0}.
Therefore we must verify if a malicious user U cannot increase the chances that
\emph{merchant M does not get paid while he has successfully revealed his winning ticket and user U remains
not punished by losing his deposit}.
Thus, the final version of properties for this protocol are as follows:

\[
\begin{tabular}{|c|c|c|c|}
\hline
property & adversary & target & \shortstack{result} \\
\hline
\lstinline{Pmin=? [F endM & balance1 >= 1]} & None & 1/2 & 1/2 \\
\hline
\lstinline{Pmax=? [F endM & balance1 >= 1]} & None & 1/2 & 1/2 \\
\hline
\lstinline{Pmax=? [F balance1 >= 1]} & M & 1/2 & 1/2 \\
\hline
\lstinline{Pmax=? [F endM & has_revealedM & balance0 >= 2 & balance1 < 1]} & U & 1/2 & \textbf{1/2} \\
\hline
\end{tabular}
\]

%\lstinputlisting{../examples/micro_v2.props}


