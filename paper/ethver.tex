\documentclass{article}

% !TEX root = ethver.tex

\usepackage{listings}
\usepackage{algpseudocode}

% Default fixed font does not support bold face
\DeclareFixedFont{\ttb}{T1}{txtt}{bx}{n}{12} % for bold
\DeclareFixedFont{\ttm}{T1}{txtt}{m}{n}{12}  % for normal

% Custom colors
\usepackage{color}
\definecolor{deepblue}{rgb}{0,0,0.5}
\definecolor{deepred}{rgb}{0.6,0,0}
\definecolor{deepgreen}{rgb}{0,0.5,0}

% Javascript style for highlighting
\newcommand\javastyle{\lstset{
language=Java,
basicstyle=\small,
otherkeywords={var, contract, uint, struct, function, bytes32, address, bool},             % Add keywords here
keywordstyle=\small\bf\color{deepblue},
emph={},          % Custom highlighting
emphstyle=\ttb\color{deepred},    % Custom highlighting style
stringstyle=\color{deepgreen},
frame=tb,                         % Any extra options here
numbers=left,
showstringspaces=false            % 
}}

% Javascript environment
\lstnewenvironment{java}[1][]
{
\javastyle
\lstset{#1}
}
{}

% Java for inline
\newcommand\javainline[1]{{\javastyle\lstinline!#1!}}



\topmargin=-15mm
\oddsidemargin=0mm
\textwidth=159.2mm
\textheight=220mm



\begin{document}

\javastyle

\tableofcontents

\section{Rock-Paper-Scissors}

%%%%%%%%%%%%%%%
\subsection{RPS v1}
%%%%%%%%%%%%%%%

%\lstinputlisting[caption=RPS v1]{../examples/rps_v1.etv}
\begin{lstlisting}
RPS v1
\end{lstlisting}


\subsubsection{Properties}

\begin{equation}
\text{\lstinputlisting{../examples/rps_v1.props}}
\label{prop:rps_v1}
\end{equation}

\subsubsection{Setup honest-honest (fairness not needed)}

\begin{enumerate}
\item ADVERSARY=-1, target = 1/3, result = 2/3
\item ADVERSARY=-1, target = 2/3, result = 2/3
\item ADVERSARY=-1, target = 1/3, result = 2/3
\item ADVERSARY=-1, target = 2/3, result = 2/3
\end{enumerate}

\subsubsection{Setup honest-adversary (fairness does not matter}

\begin{enumerate}
\item ADVERSARY=1, target = 1/3, result = 0
\item ADVERSARY=1, target = 2/3, result = 1
\item ADVERSARY=0, target = 1/3, result = 0
\item ADVERSARY=0, target = 2/3, result = 1
\end{enumerate}

\subsubsection{Debugging RPS v1}
We can start debugging RPS v1 with verifying if the final distribution of money corresponds to the chosen values 
of \lstinline{p0}, \lstinline{p1}.
PRISM provides a useful tool for generating a witness/counterexample for reachability properties.
In order to use it, we must formulate a property describing an unwanted state, e.g., 
\textit{eventually, the player 0 finishes with no money
while the chosen values were the same} (it is a draw, so he should end with \lstinline{balance = 1}):
\begin{equation}
\text{\lstinline{E [ F endA & balance0 = 0 & p0 = p1 ]}}
\label{prop:rps_v1_witness}
\end{equation}
This should never happen and in \emph{honest mode} the property indeed evaluates to false.
However in \emph{adversarial mode} (\lstinline{ADVERSARY = 1}) the property evaluates to true and PRISM
generates a \emph{witness} --- the path to reach this undesirable state.
\img{rps_v1_witness}{The witness for property (\ref{prop:rps_v1_witness})}
Figure \ref{img:rps_v1_witness} shows the witness path.
A quick look on the functions being executed in the first 20 steps gives an immediate answer, what is wrong.
Function \lstinline{finalize} is broadcast before player 0 managed to broadcast his \lstinline{player_input} function.
This is an obvious shortcoming of the contract --- broadcasting \lstinline{finalize} function should not be allowed
before both \lstinline{player_input} functions have been executed.

\paragraph{Fix 1} 

The above flaw in the contract can be easily fixed by adding an additional condition at the beginning of the 
\lstinline{finalize} function:
\begin{lstlisting}
function finalize() {
  if (num_players == 2) {
    ...
\end{lstlisting}

After this fix we can rerun the verification.
It turns out that the property (1) is still satisfied, however this
time the witness looks different (see fig. \ref{img:rps_v1a_witness}).
\img{rps_v1a_witness}{The witness for property (\ref{prop:rps_v1_witness})
run on RPS v1 model after Fix 1}
Now the alarming step in the path is \lstinline{timelock_step} which occurs before \lstinline{broadcast_player_input1}.
Basically, it means that player 1 does not execute his \lstinline{player_input} step within the given time period.
This is another obvious flaw in the contract --- the player 0 should not lose his money just because his opponent
didn't perform the required action on time.

\paragraph{Fix 2}

The shortcoming mentioned above can be overcome by adding a \emph{refund} possibility for the first player, if the second player
didn't join the game on time:
\begin{lstlisting}
function finalize() {
  if (time_elapsed > 0 && num_players == 1) {
    player_address[0].send(BET);
    ...
\end{lstlisting}

After rerunning the verification we can see that this fix also does not solve the problem completely --- the property 
(\ref{prop:rps_v1_witness}) is still satisfied.
\img{rps_v1b_witness}{The witness for the property (\ref{prop:rps_v1_witness}) run on RPS v1 model after 
Fix 1 and Fix 2}
Now the witness for the property (fig.\ \ref{img:rps_v1b_witness}) emphasizes another drawback of the protocol:
a malicious player 1 can execute the \lstinline{player_input} function twice before player 0 joins the game.
This scenario is analogous to the situation in which two players joins the game and then player 0 tries to join
the game as a \textbf{third} one.
This cannot be prevented, since even if the cautious player checks the number of joined players before his move,
there still is a possibility that two players calls the \lstinline{player_input} function simultaneously.
In such case the decision on choosing the first function call to process depends on several random factors and the player
who joins last will lose his bet while staying outside the game.

\paragraph{Fix 3}

To mitigate the last problem, we can add another \emph{refund} mechanism to the \lstinline{player_input} function
which sends the bet back to the transaction sender, if the game is already full:
\begin{lstlisting}
function player_input() {
  if (num_players >= 2) {
    msg.sender.send(msg.value);
  }
  ...
\end{lstlisting}

The complete contract code after fixes 1, 2, 3 is depicted in listing \ref{lst:rps_v2} and will be referred to as RPS v2.

%\lstinputlisting[caption=RPS v2, label=lst:rps_v2]{../examples/rps_v2.etv}
\begin{lstlisting}[caption=RPS v2, label=lst:rps_v2]
RPS v2
\end{lstlisting}

%%%%%%%%%%%%%%%
\subsection{Cryptographic commitments}
%%%%%%%%%%%%%%%

\subsubsection{Properties (similar to v3, but without committed)}

\lstinputlisting{../examples/rps_v2.props}

\subsubsection{Setup honest-honest (fairness not needed)}

\begin{enumerate}
\item ADVERSARY=-1, target = 2/3, result = 2/3
\item ADVERSARY=-1, target = 2/3, result = 2/3
\item ADVERSARY=-1, target = 2/3, result = 2/3
\item ADVERSARY=-1, target = 2/3, result = 2/3
\end{enumerate}

\subsubsection{Setup honest-adversary (fairness does not matter)}

\begin{enumerate}
\item ADVERSARY=1, target = 2/3, result = 0
\item ADVERSARY=1, target = 2/3, result = 1
\item ADVERSARY=0, target = 2/3, result = 0
\item ADVERSARY=0, target = 2/3, result = 1
\end{enumerate}

After fixes 1, 2 and 3 the property (\ref{prop:rps_v1_witness}) is no longer satisfied.
This means that the problem of lack of connection between the chosen values and the final money distribution
could potentially disappear.
However, the properties (\ref{prop:rps_v1}) still do not evaluate to the desired probabilities.

This time, the problem is more subtle.
In order to address it, we must take a closer look into how the transactions are being processed
in a blockchain-based system.
When a user calls a contract function, he creates a transaction, broadcasts it to a peer-to-peer network and 
eventually the transaction gets published in one of the upcoming blocks.
This in particular means that all the data contained in the transactions (like, e.g., arguments to function calls)
are publicly visible to all the network nodes and cannot be changed.
Thus, a malicious player in our Rock-Paper-Scissors game can wait until his opponent posts his move and 
the react with the appropriate move, which guarantees him a victory.

This feature of the blockchain-based system is reflected in a PRISM model checker in the following way.
Every variable in PRISM is \emph{public}, which means that once the value of a variable is set, it can influence
the non-deterministic choices of the next model steps.
In other words, after an honest player makes his move, its result is publicly known and the next, non-deterministic
moves of the malicious player can take advantage of this knowledge.
That is the reason why the properties (\ref{prop:rps_v1}) evaluate to 0 --- in these properties we compute the
\emph{minimal probability}, and the minimum is taken over all the non-deterministic choices in the model.

The first line of defence against such pitfalls are the cryptographic commitments.
In order to prevent the chosen value from being visible to others, the value is chosen in secret and only
the commitment is published.
In real contract, the commitment is realized using a hash function --- a value $v$ and a random salt $s$ is chosen
in secret, then the hash $h = H(v, s)$ is computed and only $h$ is published.
In EthVer, commitments are implemented under special functions \lstinline{commit} and \lstinline{open} which
postpones the random choice of a value until the \emph{open} in the PRISM model.

The Rock-Paper-Scissors code updated with the commitments is presented in listing \ref{lst:rps_v3}.

%\lstinputlisting[caption=RPS v3, label=lst:rps_v3]{../examples/rps_v3.etv}
\begin{lstlisting}[caption=RPS v3, label=lst:rps_v3]
RPS v3
\end{lstlisting}


%%%%%%%%%%%%%%%
\subsection{Misaligned incentives}
%%%%%%%%%%%%%%%

Even with cryptographic commitments implemented, the probabilities from \ref{prop:rps_v1} still evaluate to 0.
To find a bug we will once again use property (\ref{prop:rps_v1_witness}) to verify if the final distribution of money
corresponds to the chosen values of \lstinline{p0} and \lstinline{p1}.
It turns out that the property (\ref{prop:rps_v1_witness}) evaluates to true and PRISM provides us a witness 
interleaving:
\begin{itemize}
\item player 0 calls \lstinline{player_input}
\item timelock step
\item player 1 calls \lstinline{player_input}
\item player 0 calls \lstinline{finalize}
\end{itemize}
In this scenario the player 1 does not join the game within a given time period.
We have already handled this situation in Fix 2, however this fix no longer works, because now the protocol
has 2 phases: making commitments and opening commitments.

If we look carefully at the witness interleaving for RPS v3 (REFERENCE), we can see that the problem is
that the malicious player 1 can reject to call \lstinline{player_input} function on time, wait until player 0 starts
the \emph{refund procedure} and then call \lstinline{player_input} which potentially can be executed by the 
network before \lstinline{finalize}.
In such case the refund procedure from \lstinline{finalize} function fails, because at that time
\lstinline{num_players} is already equal to 2.

\paragraph{Fix 4}

One possible remedy for the problem mentioned above is to change the refund condition to: (TODO: 0/1 to variables)
\begin{lstlisting}
function finalize() {
  if (time_elapsed > 0 && committed0 && !committed1) {
    player_address[0].send(BET);
  ...
\end{lstlisting}
and prohibit players to commit after the timelock step:
\begin{lstlisting}
function player_input() {
  if (num_players >= 2 || msg.value != 1 || time_elapsed > 0) {
    msg.sender.send(msg.value);
  ...
\end{lstlisting}

After the fix the property (\ref{prop:rps_v1_witness}) is still satisfied, however this time the witness interleaving
is as follows:
\begin{itemize}
\item player 0 calls \lstinline{player_input}
\item player 1 calls \lstinline{player_input}
\item player 0 calls \lstinline{open}
\item timelock step
\item timelock step
\item player 0 calls \lstinline{finalize}
\end{itemize}

In this case both players join the game by executing the \lstinline{player_input} function and then only
player 0 opens his commitment on time.
This is a realistic scenario when a malicious player 1 waits until player 0 opens his commitment and then 
(knowing already the result of the game) refuses to open his commitment.

\paragraph{Fix 5}

To prevent from this behavior, we can introduce another refund mechanism, which transfers all the reward
to one player in case his opponent refuses to open his commitment within a given time frame (two timelock steps):
\begin{lstlisting}
function finalize() {
  if (time_elapsed > 1 && has_revealed0 && !has_revealed1) {
    player_address[0].send(reward);
  ...
\end{lstlisting}

\subsubsection{Properties (same as for v4)}

\lstinputlisting{../examples/rps_v3.props}

\subsubsection{Setup honest-honest (fairness not needed)}

\begin{enumerate}
\item ADVERSARY=-1, target = 2/3, result = 2/3
\item ADVERSARY=-1, target = 2/3, result = 2/3
\item ADVERSARY=-1, target = 2/3, result = 2/3
\item ADVERSARY=-1, target = 2/3, result = 2/3
\end{enumerate}

\subsubsection{Setup honest-adversary (fairness does not matter)}

\begin{enumerate}
\item ADVERSARY=1, target = 2/3, result = 0
\item ADVERSARY=1, target = 2/3, result = 1
\item ADVERSARY=0, target = 2/3, result = 0
\item ADVERSARY=0, target = 2/3, result = 1
\end{enumerate}

\subsubsection{Problems}

\begin{itemize}
\item Player 1 can wait until player 0 opens his commitment and in case he knows he is going to lose, he can 
avoid to open his own commitment.
\end{itemize}

\subsubsection{Solution}
Introduce a penalty: if a player does not open the commitment within some timeframe, he loses.

%%%%%%%%%%%%%%%%%%
\subsection{RPS v4}
%%%%%%%%%%%%%%%%%%
\subsubsection{Code}

%\lstinputlisting{../examples/rps_v4.etv}
\begin{lstlisting}[caption=RPS v4, label=lst:rps_v4]
RPS v4
\end{lstlisting}

\subsubsection{Properties}

\lstinputlisting{../examples/rps_v4.props}

\subsubsection{Setup honest-honest (fairness not needed)}

\begin{enumerate}
\item ADVERSARY=-1, target = 2/3, result = 2/3
\item ADVERSARY=-1, target = 2/3, result = 2/3
\item ADVERSARY=-1, target = 2/3, result = 2/3
\item ADVERSARY=-1, target = 2/3, result = 2/3
\end{enumerate}

\subsubsection{Setup honest-adversary (FAIRNESS neccesarry for P1, P3)}

\begin{enumerate}
\item ADVERSARY=1, target = 2/3, result = 2/3
\item ADVERSARY=1, target = 2/3, result = 2/3
\item ADVERSARY=0, target = 2/3, result = 2/3
\item ADVERSARY=0, target = 2/3, result = 2/3
\end{enumerate}

\subsubsection{Comments}

\begin{itemize}
\item First property (line 1) checks whether player 0 wins or draws with probability at lest 2/3 (with ADVERSARY=1)
\item Second property (line 3) checks wheter player 0 wins with probability at least 1/3 which is equivalent to player 0
loses with probability at most 2/3. (Also ADVERSARY = 1)
\item The next two properties analogously for honest player0 and ADVERSARY=0.
\end{itemize}


\section{Micropayments}

\subsection{Micropayments v1}

\subsubsection{Code}

\lstinputlisting{../examples/micro_v1.etv}

\subsubsection{Properties}

\lstinputlisting{../examples/micro_v1.props}

\subsubsection{Setup honest-adversary (fairness not needed)}

\begin{enumerate}
\item ADVERSARY=-1, target = 1/2, result = 1/2
\item ADVERSARY=-1, target = 1/2, result = 1/2
\item ADVERSARY=-1, target = 1/2, result = 1/2
\end{enumerate}

\subsubsection{Setup honest-adversary (fairness does not matter)}

\begin{enumerate}
\item ADVERSARY=-1, target = 1/2, result = 1/2
\item ADVERSARY=0, target = 1/2, result = 1
\item ADVERSARY=1, target = 1/2, \textbf{result = 1/2 (ok?)}
\end{enumerate}

\subsection{Micropayments v2}

\subsubsection{Code}

\lstinputlisting{../examples/micro_v2.etv}

\subsubsection{Properties}

\lstinputlisting{../examples/micro_v2.props}


\subsubsection{Setup honest-adversary (fairness not needed)}

\begin{enumerate}
\item ADVERSARY=-1, target = 1/2, result = 1/2
\item ADVERSARY=-1, target = 1/2, result = 1/2
\item ADVERSARY=-1, target = 1/2, result = 1/2
\end{enumerate}

\subsubsection{Setup honest-adversary (fairness does not matter)}

\begin{enumerate}
\item ADVERSARY=-1, target = 1/2, result = 1/2
\item ADVERSARY=0, target = 1/2, result = 1/2
\item ADVERSARY=1, target = 1/2, result = 1/2
\end{enumerate}

\end{document}