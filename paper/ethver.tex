\documentclass{article}

% !TEX root = ethver.tex

\usepackage{listings}
\usepackage{algpseudocode}

% Default fixed font does not support bold face
\DeclareFixedFont{\ttb}{T1}{txtt}{bx}{n}{12} % for bold
\DeclareFixedFont{\ttm}{T1}{txtt}{m}{n}{12}  % for normal

% Custom colors
\usepackage{color}
\definecolor{deepblue}{rgb}{0,0,0.5}
\definecolor{deepred}{rgb}{0.6,0,0}
\definecolor{deepgreen}{rgb}{0,0.5,0}

% Javascript style for highlighting
\newcommand\javastyle{\lstset{
language=Java,
basicstyle=\small,
otherkeywords={var, contract, uint, struct, function, bytes32, address, bool},             % Add keywords here
keywordstyle=\small\bf\color{deepblue},
emph={},          % Custom highlighting
emphstyle=\ttb\color{deepred},    % Custom highlighting style
stringstyle=\color{deepgreen},
frame=tb,                         % Any extra options here
numbers=left,
showstringspaces=false            % 
}}

% Javascript environment
\lstnewenvironment{java}[1][]
{
\javastyle
\lstset{#1}
}
{}

% Java for inline
\newcommand\javainline[1]{{\javastyle\lstinline!#1!}}



\topmargin=-15mm
\oddsidemargin=0mm
\textwidth=159.2mm
\textheight=220mm



\begin{document}

\javastyle

\tableofcontents

\section{Rock-Paper-Scissors}

%%%%%%%%%%%%%%%
\subsection{RPS v1}
%%%%%%%%%%%%%%%

\lstinputlisting[caption=RPS v1]{../examples/rps_v1.etv}

\subsubsection{Properties}

\lstinputlisting{../examples/rps_v1.props}

\subsubsection{Setup honest-honest (fairness not needed)}

\begin{enumerate}
\item ADVERSARY=-1, target = 1/3, result = 2/3
\item ADVERSARY=-1, target = 2/3, result = 2/3
\item ADVERSARY=-1, target = 1/3, result = 2/3
\item ADVERSARY=-1, target = 2/3, result = 2/3
\end{enumerate}

\subsubsection{Setup honest-adversary (fairness does not matter}

\begin{enumerate}
\item ADVERSARY=1, target = 1/3, result = 0
\item ADVERSARY=1, target = 2/3, result = 1
\item ADVERSARY=0, target = 1/3, result = 0
\item ADVERSARY=0, target = 2/3, result = 1
\end{enumerate}

\subsubsection{Debugging RPS v1}
We can start debugging RPS v1 with verifying if the final distribution of money corresponds to the chosen values 
of \lstinline{p0}, \lstinline{p1}.
PRISM provides a useful tool for generating a witness/counterexample for reachability properties.
In order to use it, we must formulate a property describing an unwanted state, e.g., 
\textit{eventually, the player 0 finishes with no money
while the chosen values were the same} (it is a draw, so he should end with \lstinline{balance = 1}):
\begin{equation}
\text{\lstinline{E [ F endA & balance0 = 0 & p0 = 1 & p1 = 1 ]}}
\label{prop:rps_v1_witness}
\end{equation}
This should never happen and in \emph{honest mode} the property indeed evaluates to false.
However in \emph{adversarial mode} (\lstinline{ADVERSARY = 1}) the property evaluates to true and PRISM
generates a \emph{witness} --- the path to reach this undesirable state.
\img{rps_v1_witness}{The witness for property \ref{prop:rps_v1_witness}}
Figure \ref{img:rps_v1_witness} shows the witness path.
A quick look on the functions being executed in the first 20 steps gives an immediate answer, what is wrong.
Function \lstinline{finalize} is broadcast before player 0 managed to broadcast his \lstinline{player_input} function.
This is an obvious shortcoming of the contract --- broadcasting \lstinline{finalize} function should not be allowed
before both \lstinline{player_input} functions have been executed.

\paragraph{Fix 1} 

The above flaw in the contract can be easily fixed by adding an additional condition at the beginning of the 
\lstinline{finalize} function:
\begin{lstlisting}
function finalize() {
  if (num_players == 2) {
    ...
\end{lstlisting}

After this fix we can rerun the verification.
It turns out that the property (1) is still satisfied, however this
time the witness looks different (see fig. \ref{img:rps_v1a_witness}).
\img{rps_v1a_witness}{The witness for property \lstinline{E [ F endA & balance0 = 0 & p0 = 1 & p1 = 1 ]}
run on RPS v1 model after adding Fix 1 (the requirement for executing both \lstinline{player_input} functions before
execution of \lstinline{finalize})}
Now the alarming step in the path is \lstinline{timelock_step} which occurs before \lstinline{broadcast_player_input1}.
Basically, it means that the player 1 does not execute his \lstinline{player_input} step within the given time period.
This is another obvious flaw in the contract --- the player 0 should not lose his money just because his opponent
didn't perform the required action on time.

\paragraph{Fix 2}

The shortcoming mentioned above can be overcome by adding a \emph{refund} possibility for the first player, if the second player
didn't join the game on time:
\begin{lstlisting}
function finalize() {
  if (time_elapsed > 0 && num_players == 1) {
    player_address[0].send(BET);
    ...
\end{lstlisting}

After rerunning the verification we can see that this fix also does not solve the problem completely --- the property 
(\ref{prop:rps_v1_witness}) is still not satisfied.
\img{rps_v1b_witness}{The witness for property (\ref{prop:rps_v1_witness}) run on RPS v1 model after adding 
Fix 1 and Fix 2}
Now the witness for the property (fig. \ref{img:rps_v1b_witness}) emphasizes another drawback of the protocol:
a malicious player 1 can execute the \lstinline{player_input} function twice before player 0 joins the game.
This scenario is analogous to the situation in which two players joins the game and then player 0 tries to join
the game as a \textbf{third} one.
This cannot be prevented, since even if the cautious player checks the number of joined players before his move,
there still is a possibility that two players calls the \lstinline{player_input} function simultaneously.
In such case the decision on choosing the first function call to process depends on several random factors and the player
who joins last will lose his bet while staying outside the game.

\paragraph{Fix 3}

To mitigate the last problem, one can add another \emph{refund} mechanism to the \lstinline{player_input} function
which sends the bet back to the transaction sender, if the game is already full:
\begin{lstlisting}
function player_input() {
  if (num_players >= 2) {
    msg.sender.send(msg.value);
  }
  ...
\end{lstlisting}

The complete contract code after fixes 1, 2, 3 is depicted in listing \ref{lst:rps_v2}.
\lstinputlisting[caption=RPS v2, label=lst:rps_v2]{../examples/rps_v2.etv}
  

%%%%%%%%%%%%%%%
\subsection{RPS v2}
%%%%%%%%%%%%%%%

\lstinputlisting{../examples/rps_v2.etv}

\subsubsection{Properties (similar to v3, but without committed)}

\lstinputlisting{../examples/rps_v2.props}

\subsubsection{Setup honest-honest (fairness not needed)}

\begin{enumerate}
\item ADVERSARY=-1, target = 2/3, result = 2/3
\item ADVERSARY=-1, target = 2/3, result = 2/3
\item ADVERSARY=-1, target = 2/3, result = 2/3
\item ADVERSARY=-1, target = 2/3, result = 2/3
\end{enumerate}

\subsubsection{Setup honest-adversary (fairness does not matter)}

\begin{enumerate}
\item ADVERSARY=1, target = 2/3, result = 0
\item ADVERSARY=1, target = 2/3, result = 1
\item ADVERSARY=0, target = 2/3, result = 0
\item ADVERSARY=0, target = 2/3, result = 1
\end{enumerate}

%%%%%%%%%%%%%%%
\subsection{RPS v3}
%%%%%%%%%%%%%%%

\subsubsection{Code}

\lstinputlisting{../examples/rps_v3.etv}

\subsubsection{Properties (same as for v4)}

\lstinputlisting{../examples/rps_v3.props}

\subsubsection{Setup honest-honest (fairness not needed)}

\begin{enumerate}
\item ADVERSARY=-1, target = 2/3, result = 2/3
\item ADVERSARY=-1, target = 2/3, result = 2/3
\item ADVERSARY=-1, target = 2/3, result = 2/3
\item ADVERSARY=-1, target = 2/3, result = 2/3
\end{enumerate}

\subsubsection{Setup honest-adversary (fairness does not matter)}

\begin{enumerate}
\item ADVERSARY=1, target = 2/3, result = 0
\item ADVERSARY=1, target = 2/3, result = 1
\item ADVERSARY=0, target = 2/3, result = 0
\item ADVERSARY=0, target = 2/3, result = 1
\end{enumerate}

\subsubsection{Problems}

\begin{itemize}
\item Player 1 can wait until player 0 opens his commitment and in case he knows he is going to lose, he can 
avoid to open his own commitment.
\end{itemize}

\subsubsection{Solution}
Introduce a penalty: if a player does not open the commitment within some timeframe, he loses.

%%%%%%%%%%%%%%%%%%
\subsection{RPS v4}
%%%%%%%%%%%%%%%%%%
\subsubsection{Code}

\lstinputlisting{../examples/rps_v4.etv}

\subsubsection{Properties}

\lstinputlisting{../examples/rps_v4.props}

\subsubsection{Setup honest-honest (fairness not needed)}

\begin{enumerate}
\item ADVERSARY=-1, target = 2/3, result = 2/3
\item ADVERSARY=-1, target = 2/3, result = 2/3
\item ADVERSARY=-1, target = 2/3, result = 2/3
\item ADVERSARY=-1, target = 2/3, result = 2/3
\end{enumerate}

\subsubsection{Setup honest-adversary (FAIRNESS neccesarry for P1, P3)}

\begin{enumerate}
\item ADVERSARY=1, target = 2/3, result = 2/3
\item ADVERSARY=1, target = 2/3, result = 2/3
\item ADVERSARY=0, target = 2/3, result = 2/3
\item ADVERSARY=0, target = 2/3, result = 2/3
\end{enumerate}

\subsubsection{Comments}

\begin{itemize}
\item First property (line 1) checks whether player 0 wins or draws with probability at lest 2/3 (with ADVERSARY=1)
\item Second property (line 3) checks wheter player 0 wins with probability at least 1/3 which is equivalent to player 0
loses with probability at most 2/3. (Also ADVERSARY = 1)
\item The next two properties analogously for honest player0 and ADVERSARY=0.
\end{itemize}


\section{Micropayments}

\subsection{Micropayments v1}

\subsubsection{Code}

\lstinputlisting{../examples/micro_v1.etv}

\subsubsection{Properties}

\lstinputlisting{../examples/micro_v1.props}

\subsubsection{Setup honest-adversary (fairness not needed)}

\begin{enumerate}
\item ADVERSARY=-1, target = 1/2, result = 1/2
\item ADVERSARY=-1, target = 1/2, result = 1/2
\item ADVERSARY=-1, target = 1/2, result = 1/2
\end{enumerate}

\subsubsection{Setup honest-adversary (fairness does not matter)}

\begin{enumerate}
\item ADVERSARY=-1, target = 1/2, result = 1/2
\item ADVERSARY=0, target = 1/2, result = 1
\item ADVERSARY=1, target = 1/2, \textbf{result = 1/2 (ok?)}
\end{enumerate}

\subsection{Micropayments v2}

\subsubsection{Code}

\lstinputlisting{../examples/micro_v2.etv}

\subsubsection{Properties}

\lstinputlisting{../examples/micro_v2.props}


\subsubsection{Setup honest-adversary (fairness not needed)}

\begin{enumerate}
\item ADVERSARY=-1, target = 1/2, result = 1/2
\item ADVERSARY=-1, target = 1/2, result = 1/2
\item ADVERSARY=-1, target = 1/2, result = 1/2
\end{enumerate}

\subsubsection{Setup honest-adversary (fairness does not matter)}

\begin{enumerate}
\item ADVERSARY=-1, target = 1/2, result = 1/2
\item ADVERSARY=0, target = 1/2, result = 1/2
\item ADVERSARY=1, target = 1/2, result = 1/2
\end{enumerate}

\end{document}